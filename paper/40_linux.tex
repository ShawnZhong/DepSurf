

\section{Stability of eBPF Programs}
\label{sec:stability}


\begin{table*}[t]
  \centering
  \newcommand{\thead}[3]{\multicolumn{#1}{#2}{\bfseries #3}}
  \newcommand{\mcol}[2]{\multicolumn{#1}{r||}{#2}}
  \newcommand{\chead}[2]{\multirow{#1}{*}{\rotatebox[origin=c]{90}{\bfseries #2}}}
  \newcommand{\ver}[2]{#1 $\rightarrow$ #2}

  % \setlength{\tabcolsep}{5pt}

  \begin{tabular}{c|l||r|r|r|r|r|l}
     & \thead{1}{r||}{Distribution}
    %  & \thead{5}{c||}{Ubuntu 18.04}
     & \thead{5}{c|}{Ubuntu 20.04}
     & \thead{1}{c}{Error}
    \\ \hline


     & \mcol{1}{Versions}           & \ver{5.4}{5.8} & \ver{5.8}{5.11} & \ver{5.11}{5.13} & \ver{5.13}{5.15} & Total         &          \\ \hline\hline

    \chead{7}{Struct}
     & Added                        & 529 ( 6.3\%)   & 513 ( 5.9\%)    & 233 ( 2.6\%)     & 283 ( 3.1\%)     & 1426 (17.0\%) & Compiler \\
     & Removed                      & 203 ( 2.4\%)   & 156 ( 1.8\%)    & 96 ( 1.1\%)      & 117 ( 1.3\%)     & 440 ( 5.2\%)  & Compiler \\
     & Changed                      & 973 (11.6\%)   & 771 ( 8.8\%)    & 476 ( 5.2\%)     & 656 ( 7.1\%)     & 1513 (18.0\%)            \\
     & - Field added                & 485 ( 5.8\%)   & 439 ( 5.0\%)    & 272 ( 3.0\%)     & 365 ( 4.0\%)     & 944 (11.2\%)  & Compiler \\
     & - Field removed              & 315 ( 3.8\%)   & 232 ( 2.7\%)    & 154 ( 1.7\%)     & 163 ( 1.8\%)     & 533 ( 6.3\%)  & Compiler \\
     & - Field type changed         & 219 ( 2.6\%)   & 164 ( 1.9\%)    & 111 ( 1.2\%)     & 140 ( 1.5\%)     & 400 ( 4.8\%)  & Silent   \\
     & - Layout changed             & 897 (10.7\%)   & 720 ( 8.3\%)    & 438 ( 4.8\%)     & 605 ( 6.6\%)     & 1408 (16.8\%)            \\
    \hline
    \chead{8}{Function}
     & Added                        & 3564 ( 7.4\%)  & 3080 ( 6.2\%)   & 1343 ( 2.6\%)    & 2121 ( 4.1\%)    & 8958 (18.6\%) & Hook     \\
     & Removed                      & 2000 ( 4.2\%)  & 1076 ( 2.2\%)   & 755 ( 1.5\%)     & 1123 ( 2.1\%)    & 3804 ( 7.9\%) & Hook     \\
     & Changed                      & 932 ( 1.9\%)   & 1176 ( 2.4\%)   & 549 ( 1.1\%)     & 640 ( 1.2\%)     & 2385 ( 5.0\%)            \\
     & - Param added                & 535 ( 1.1\%)   & 375 ( 0.8\%)    & 428 ( 0.8\%)     & 298 ( 0.6\%)     & 1194 ( 2.5\%) & Silent   \\
     & - Param removed              & 475 ( 1.0\%)   & 343 ( 0.7\%)    & 172 ( 0.3\%)     & 194 ( 0.4\%)     & 848 ( 1.8\%)  & Silent   \\
     & - Param type changed         & 208 ( 0.4\%)   & 623 ( 1.3\%)    & 42 ( 0.1\%)      & 214 ( 0.4\%)     & 799 ( 1.7\%)  & Silent   \\
     & - Param reordered            & 132 ( 0.3\%)   & 163 ( 0.3\%)    & 280 ( 0.5\%)     & 104 ( 0.2\%)     & 544 ( 1.1\%)  & Silent   \\
     & - Return type changed        & 130 ( 0.3\%)   & 82 ( 0.2\%)     & 51 ( 0.1\%)      & 111 ( 0.2\%)     & 276 ( 0.6\%)  & Silent   \\
    \hline
    \chead{7}{Tracepoint Event}
     & Added                        & 51 (11.0\%)    & 31 ( 6.2\%)     & 13 ( 2.5\%)      & 17 ( 3.3\%)      & 94 (20.2\%)   & Compiler \\
     & Removed                      & 16 ( 3.4\%)    & 6 ( 1.2\%)      & 15 ( 2.9\%)      & 15 ( 2.9\%)      & 34 ( 7.3\%)   & Compiler \\
     & Changed                      & 36 ( 7.7\%)    & 14 ( 2.8\%)     & 10 ( 1.9\%)      & 17 ( 3.3\%)      & 47 (10.1\%)              \\
     & - Field added                & 8 ( 1.7\%)     & 11 ( 2.2\%)     & 6 ( 1.1\%)       & 6 ( 1.1\%)       & 20 ( 4.3\%)   & Compiler \\
     & - Field removed              & 8 ( 1.7\%)     & 3 ( 0.6\%)      & 1 ( 0.2\%)       & 2 ( 0.4\%)       & 11 ( 2.4\%)   & Compiler \\
     & - Field type changed         & 25 ( 5.4\%)    & 0 ( 0.0\%)      & 5 ( 1.0\%)       & 10 ( 1.9\%)      & 21 ( 4.5\%)   & Silent   \\
     & - Layout changed             & 31 ( 6.7\%)    & 14 ( 2.8\%)     & 9 ( 1.7\%)       & 14 ( 2.7\%)      & 43 ( 9.2\%)              \\
    \hline
    \chead{8}{LSM Hook}
     & Added                        & 10 ( 5.3\%)    & 1 ( 0.5\%)      & 5 ( 2.5\%)       & 0 ( 0.0\%)       & 16 ( 8.5\%)   & Hook     \\
     & Removed                      & 1 ( 0.5\%)     & 0 ( 0.0\%)      & 1 ( 0.5\%)       & 0 ( 0.0\%)       & 2 ( 1.1\%)    & Hook     \\
     & Changed                      & 18 ( 9.5\%)    & 5 ( 2.5\%)      & 4 ( 2.0\%)       & 5 ( 2.5\%)       & 23 (12.2\%)              \\
     & - Param added                & 13 ( 6.9\%)    & 4 ( 2.0\%)      & 4 ( 2.0\%)       & 0 ( 0.0\%)       & 20 (10.6\%)   & Silent   \\
     & - Param removed              & 13 ( 6.9\%)    & 2 ( 1.0\%)      & 0 ( 0.0\%)       & 1 ( 0.5\%)       & 15 ( 7.9\%)   & Silent   \\
     & - Param type changed         & 6 ( 3.2\%)     & 1 ( 0.5\%)      & 0 ( 0.0\%)       & 4 ( 2.0\%)       & 3 ( 1.6\%)    & Silent   \\
     & - Param reordered            & 0 ( 0.0\%)     & 0 ( 0.0\%)      & 4 ( 2.0\%)       & 0 ( 0.0\%)       & 4 ( 2.1\%)    & Silent   \\
  \end{tabular}
  \caption{Linux kernel source code changes}
  \label{tab:kernel_source_code_changes}
\end{table*}

The Linux kernel is a large and complex system, and it is constantly evolving.
The kernel source code is updated frequently, and the changes can affect the eBPF programs.
In this section, we analyze the stability of eBPF programs by examining the changes in the kernel source code.
We compare the changes in the kernel source code between different versions and examine how the changes affect the eBPF programs.


\minisec{Stable ABI of eBPF}
The Linux kernel provides some guarantees about the stability of eBPF ABI (Application Binary Interface).
The eBPF ISA (Instruction Set Architecture) and the calling convention of eBPF programs are considered stable,
and the kernel developers are committed to maintaining backward compatibility~\cite{bpf-design, bpf-isa}.
In addition to the low-level infrastructure, the definition of helper functions, arguments to the eBPF programs,
and the recognized return values are also part of the stable ABI~\cite{bpf-design}.
There are also attempts to standardize the format of the object file~\cite{bpf-elf} to enable the distribution of compiled eBPF programs.

\minisec{Unstable ABI of eBPF}
Despite all the efforts to maintain the stability of the eBPF ABI, the eBPF programs are still susceptible to changes in the kernel source code.
First, the hooks for eBPF programs are not part of the stable ABI.
This is especially true for dynamic instrumentation using kprobe,
where the hooks are tens of thousands of kernel functions and are subject to frequent additions, removals, and changes.
Therefore a kprobe that can be attached to a function in one version of the kernel may not be available in another version.
For static instrumentation, tracepoints are often mistakenly considered stable~\cite{tcp-tracepoint, bpf-book}, but they are also subject to changes.

In addition to the hooks, the content of the context argument passed to the eBPF programs is also unstable.
For eBPF programs taking in \texttt{struct pt\_regs *} as the context argument (e.g., kprobe, uprobes, and LSM programs),
the content in the \texttt{struct pt\_regs *} are merely a snapshot of the CPU registers at the time of the event.
eBPF programs read the content of the registers as the arguments to the function,
but the function signature can change.
Therefore the eBPF programs that read the arguments of the function are also unstable.
If the kernel developers change the function signature, the eBPF programs can silently read the wrong arguments.


Finally, the eBPF programs that interact with the kernel data structures are also unstable.
The helper function, \texttt{bpf\_probe\_read\_kernel},
which allows eBPF programs to read arbitrary memory locations in the kernel,
is available for all types of eBPF programs.
It is often used to read a kernel data structure to a local buffer or struct in the eBPF program.
Since kernel data structures constantly evolve, the eBPF programs may fail to read the expected data or read the wrong data.

\minisec{CO-RE and BTF}
\shawn{Let's see if we need to mention CO-RE and BTF here.
  We can just talk about compatibility/stability in the context of the source code
  (assuming that the eBPF programs are compiled with the same version of the kernel source code), as opposed to the binary compatibility.
}

\minisec{Program Types}
Based on the three sources of instability above, we assess the stability of each type of eBPF program,
and the result is summarized in Table~\ref{tab:ebpf_program_types}.
Network programs are the most stable since they are attached to the well-defined hooks in the network stack,
and the context argument points to packet data, which is relatively stable.
Helper functions that read kernel memory are generally not used in network programs, although there is no restriction.

Trace programs attached to kprobes are the least stable, for all three sources of instability.
Programs attached to tracepoints are relatively more stable, since the tracepoints are better maintained to try to avoid breaking changes,
and the context argument is a well-defined event struct, but such programs may still read internal kernel data structures.
eBPF programs attached to LSM hooks are more stable than kprobes due to fewer well-maintained hooks, but still less stable than tracepoints.


\minisec{Linux Source Code Evolution}
To understand the stability of eBPF programs, we examine the changes in the Linux kernel source code.
We use the popular Linux distribution, Ubuntu, as the reference, and compare all Linux kernel versions officially supported by Ubuntu 18.04 and 20.04.
For all supported kernel versions, we extract the debug info from the compiled kernel and compare how the struct and function definitions have changed.
The result of the comparison is summarized in Table~\ref{tab:kernel_source_code_changes}.



\todo{Explain the result in the table.}

\minisec{Consequence}
\todo{Explain the consequence of the instability, and give an example with LSM}
